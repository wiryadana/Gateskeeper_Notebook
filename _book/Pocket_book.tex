% Options for packages loaded elsewhere
\PassOptionsToPackage{unicode}{hyperref}
\PassOptionsToPackage{hyphens}{url}
%
\documentclass[
]{book}
\usepackage{amsmath,amssymb}
\usepackage{lmodern}
\usepackage{ifxetex,ifluatex}
\ifnum 0\ifxetex 1\fi\ifluatex 1\fi=0 % if pdftex
  \usepackage[T1]{fontenc}
  \usepackage[utf8]{inputenc}
  \usepackage{textcomp} % provide euro and other symbols
\else % if luatex or xetex
  \usepackage{unicode-math}
  \defaultfontfeatures{Scale=MatchLowercase}
  \defaultfontfeatures[\rmfamily]{Ligatures=TeX,Scale=1}
\fi
% Use upquote if available, for straight quotes in verbatim environments
\IfFileExists{upquote.sty}{\usepackage{upquote}}{}
\IfFileExists{microtype.sty}{% use microtype if available
  \usepackage[]{microtype}
  \UseMicrotypeSet[protrusion]{basicmath} % disable protrusion for tt fonts
}{}
\makeatletter
\@ifundefined{KOMAClassName}{% if non-KOMA class
  \IfFileExists{parskip.sty}{%
    \usepackage{parskip}
  }{% else
    \setlength{\parindent}{0pt}
    \setlength{\parskip}{6pt plus 2pt minus 1pt}}
}{% if KOMA class
  \KOMAoptions{parskip=half}}
\makeatother
\usepackage{xcolor}
\IfFileExists{xurl.sty}{\usepackage{xurl}}{} % add URL line breaks if available
\IfFileExists{bookmark.sty}{\usepackage{bookmark}}{\usepackage{hyperref}}
\hypersetup{
  pdftitle={The Gatekeeper's Notebook},
  pdfauthor={Kadek Adit Wiryadana},
  hidelinks,
  pdfcreator={LaTeX via pandoc}}
\urlstyle{same} % disable monospaced font for URLs
\usepackage{color}
\usepackage{fancyvrb}
\newcommand{\VerbBar}{|}
\newcommand{\VERB}{\Verb[commandchars=\\\{\}]}
\DefineVerbatimEnvironment{Highlighting}{Verbatim}{commandchars=\\\{\}}
% Add ',fontsize=\small' for more characters per line
\usepackage{framed}
\definecolor{shadecolor}{RGB}{248,248,248}
\newenvironment{Shaded}{\begin{snugshade}}{\end{snugshade}}
\newcommand{\AlertTok}[1]{\textcolor[rgb]{0.94,0.16,0.16}{#1}}
\newcommand{\AnnotationTok}[1]{\textcolor[rgb]{0.56,0.35,0.01}{\textbf{\textit{#1}}}}
\newcommand{\AttributeTok}[1]{\textcolor[rgb]{0.77,0.63,0.00}{#1}}
\newcommand{\BaseNTok}[1]{\textcolor[rgb]{0.00,0.00,0.81}{#1}}
\newcommand{\BuiltInTok}[1]{#1}
\newcommand{\CharTok}[1]{\textcolor[rgb]{0.31,0.60,0.02}{#1}}
\newcommand{\CommentTok}[1]{\textcolor[rgb]{0.56,0.35,0.01}{\textit{#1}}}
\newcommand{\CommentVarTok}[1]{\textcolor[rgb]{0.56,0.35,0.01}{\textbf{\textit{#1}}}}
\newcommand{\ConstantTok}[1]{\textcolor[rgb]{0.00,0.00,0.00}{#1}}
\newcommand{\ControlFlowTok}[1]{\textcolor[rgb]{0.13,0.29,0.53}{\textbf{#1}}}
\newcommand{\DataTypeTok}[1]{\textcolor[rgb]{0.13,0.29,0.53}{#1}}
\newcommand{\DecValTok}[1]{\textcolor[rgb]{0.00,0.00,0.81}{#1}}
\newcommand{\DocumentationTok}[1]{\textcolor[rgb]{0.56,0.35,0.01}{\textbf{\textit{#1}}}}
\newcommand{\ErrorTok}[1]{\textcolor[rgb]{0.64,0.00,0.00}{\textbf{#1}}}
\newcommand{\ExtensionTok}[1]{#1}
\newcommand{\FloatTok}[1]{\textcolor[rgb]{0.00,0.00,0.81}{#1}}
\newcommand{\FunctionTok}[1]{\textcolor[rgb]{0.00,0.00,0.00}{#1}}
\newcommand{\ImportTok}[1]{#1}
\newcommand{\InformationTok}[1]{\textcolor[rgb]{0.56,0.35,0.01}{\textbf{\textit{#1}}}}
\newcommand{\KeywordTok}[1]{\textcolor[rgb]{0.13,0.29,0.53}{\textbf{#1}}}
\newcommand{\NormalTok}[1]{#1}
\newcommand{\OperatorTok}[1]{\textcolor[rgb]{0.81,0.36,0.00}{\textbf{#1}}}
\newcommand{\OtherTok}[1]{\textcolor[rgb]{0.56,0.35,0.01}{#1}}
\newcommand{\PreprocessorTok}[1]{\textcolor[rgb]{0.56,0.35,0.01}{\textit{#1}}}
\newcommand{\RegionMarkerTok}[1]{#1}
\newcommand{\SpecialCharTok}[1]{\textcolor[rgb]{0.00,0.00,0.00}{#1}}
\newcommand{\SpecialStringTok}[1]{\textcolor[rgb]{0.31,0.60,0.02}{#1}}
\newcommand{\StringTok}[1]{\textcolor[rgb]{0.31,0.60,0.02}{#1}}
\newcommand{\VariableTok}[1]{\textcolor[rgb]{0.00,0.00,0.00}{#1}}
\newcommand{\VerbatimStringTok}[1]{\textcolor[rgb]{0.31,0.60,0.02}{#1}}
\newcommand{\WarningTok}[1]{\textcolor[rgb]{0.56,0.35,0.01}{\textbf{\textit{#1}}}}
\usepackage{longtable,booktabs,array}
\usepackage{calc} % for calculating minipage widths
% Correct order of tables after \paragraph or \subparagraph
\usepackage{etoolbox}
\makeatletter
\patchcmd\longtable{\par}{\if@noskipsec\mbox{}\fi\par}{}{}
\makeatother
% Allow footnotes in longtable head/foot
\IfFileExists{footnotehyper.sty}{\usepackage{footnotehyper}}{\usepackage{footnote}}
\makesavenoteenv{longtable}
\usepackage{graphicx}
\makeatletter
\def\maxwidth{\ifdim\Gin@nat@width>\linewidth\linewidth\else\Gin@nat@width\fi}
\def\maxheight{\ifdim\Gin@nat@height>\textheight\textheight\else\Gin@nat@height\fi}
\makeatother
% Scale images if necessary, so that they will not overflow the page
% margins by default, and it is still possible to overwrite the defaults
% using explicit options in \includegraphics[width, height, ...]{}
\setkeys{Gin}{width=\maxwidth,height=\maxheight,keepaspectratio}
% Set default figure placement to htbp
\makeatletter
\def\fps@figure{htbp}
\makeatother
\setlength{\emergencystretch}{3em} % prevent overfull lines
\providecommand{\tightlist}{%
  \setlength{\itemsep}{0pt}\setlength{\parskip}{0pt}}
\setcounter{secnumdepth}{5}
\usepackage{booktabs}
\ifluatex
  \usepackage{selnolig}  % disable illegal ligatures
\fi
\usepackage[]{natbib}
\bibliographystyle{apalike}

\title{The Gatekeeper's Notebook}
\author{Kadek Adit Wiryadana}
\date{2021-11-13}

\begin{document}
\maketitle

{
\setcounter{tocdepth}{1}
\tableofcontents
}
\hypertarget{gatekeepers-notebook}{%
\chapter{Gatekeepers Notebook}\label{gatekeepers-notebook}}

\begin{Shaded}
\begin{Highlighting}[]
\NormalTok{knitr}\SpecialCharTok{::}\FunctionTok{include\_graphics}\NormalTok{(}\StringTok{"COVER.png"}\NormalTok{)}
\end{Highlighting}
\end{Shaded}

\begin{center}\includegraphics[width=7.11in]{COVER} \end{center}

\hypertarget{sepatah-kata}{%
\chapter{Sepatah Kata}\label{sepatah-kata}}

Buku ini disusun untuk dokter Jaga Unit Gawat Darurat. Buku berisikan panduan umum berdasarkan pengalaman penulis beserta hasil diskusi dengan teman sejawat dan konsultasi dengan dokter Spesialis dan Managemen.

\hypertarget{panduan}{%
\section{Panduan}\label{panduan}}

Buku ini hanya berisikan panduan umum. Buku tidak disusun dengan tujuan menjadi panduan komprehensif ataupun Cheatsheet. Tetap sesuaikan dengan kondisi dan selalu dengarkan kata hati.

\hypertarget{narahubung}{%
\section{Narahubung}\label{narahubung}}

Buku disusun oleh dr. Kadek Adit Wiryadana, pertanyaan dan saran dapat disampaikan pada \href{mailto:ka.wiryadana@gmail.com}{\nolinkurl{ka.wiryadana@gmail.com}}

\hypertarget{peraturan-kerja}{%
\chapter{Peraturan Kerja}\label{peraturan-kerja}}

Hal-hal yang perlu dipatuhi saat bekerja

\hypertarget{jam-kerja}{%
\section{Jam Kerja}\label{jam-kerja}}

Jam kerja dibagi menjadi tiga shift.

\begin{itemize}
\item
  Pagi (08.00 - 14.00 Wita)
\item
  Sore (14.00 - 20.00 Wita)
\item
  Malam (20.00 - 08.00 Wita)
\end{itemize}

\hypertarget{poin-kerja}{%
\section{Poin Kerja}\label{poin-kerja}}

\hypertarget{khusus-fultime}{%
\subsection*{khusus Fultime}\label{khusus-fultime}}
\addcontentsline{toc}{subsection}{khusus Fultime}

Sesuai dengan kontrak, jaga dihitung dengan poin dengan rincian:

\begin{itemize}
\item
  Jaga Pagi \textasciitilde{} 1 poin
\item
  Jaga Sore \textasciitilde{} 1 Poin
\item
  Jaga Malam \textasciitilde{} 2 Poin
\end{itemize}

Dokter kontrak dalam sebulan minimal mendapatkan 24 poin.

\hypertarget{poin-kerja-part-time}{%
\subsection*{Poin Kerja Part-time}\label{poin-kerja-part-time}}
\addcontentsline{toc}{subsection}{Poin Kerja Part-time}

Dokter Part-time diberikan kesempatan untuk memilih waktu jaga, jika ada shift dimana tidak bisa diisi oleh dokter fulltime.

\hypertarget{pakaian-kerja}{%
\section{Pakaian Kerja}\label{pakaian-kerja}}

\begin{itemize}
\tightlist
\item
  Saat datang ke Rumah Sakit diharapkan berpakaian Rapi.
\item
  Saat bekerja menggunakan set baju jaga yang telah disediakan (baju, celana, headcap, alas kaki), kecuali ukuran baju jaga tidak cukup
\item
  Menggunakan masker N95 atau KN95
\item
  Menggunakan baju operasi (gown)
\end{itemize}

\hypertarget{cross}{%
\chapter{Alur Kerja}\label{cross}}

\hypertarget{pasien-baru-ugd}{%
\section{Pasien Baru UGD}\label{pasien-baru-ugd}}

\begin{enumerate}
\def\labelenumi{\arabic{enumi}.}
\tightlist
\item
  Pasien datang ke UGD
\item
  Tanyakan keluhan utama
\item
  Screening TTV dan SpO2
\item
  Lakukan pemeriksaan anamnesis dan pemeriksaan fisik
\item
  KIE pemeriksaan penunjang untuk penegakan diagnosis dan KIE waktu tunggu
\item
  Berikan pengobatan awal dan lakukan pemeriksaan penunjang
\item
  Konsul ke dr Spesialis terkait jika data cukup
\item
  Kie pasien dan keluarga terkait diagnosis dan pengobatan yang akan diberikan
\item
  Berikan pengobatan.
\end{enumerate}

\hypertarget{pasien-rujukan-poli}{%
\section{Pasien rujukan Poli}\label{pasien-rujukan-poli}}

\begin{enumerate}
\def\labelenumi{\arabic{enumi}.}
\tightlist
\item
  Pasien diantar perawat poli ke UGD.
\item
  Operan dan baca pengantar rawat inap dari dr Spesialis.
\item
  Lakukan pemeriksaan anamnesis dan pemeriksaan fisik pasien
\item
  Konfirmasi instruksi dokter spesialis jika masih ada yang ragu atau kurang.
\item
  Lakukan pemeriksaan yang diinstruksikan dr Spesialis
\item
  Berikan pengobatan yang diinstruksikan dr Spesialis
\item
  Konsulkan hasil pemeriksaan penunjang ke dr Spesialis
\end{enumerate}

\hypertarget{pasien-pre-op}{%
\section{Pasien Pre-OP}\label{pasien-pre-op}}

\begin{enumerate}
\def\labelenumi{\arabic{enumi}.}
\tightlist
\item
  Pasien diantar keluarga ke UGD
\item
  Tanyakan keluhan utama, jika dikatakan rencana operasi.
\item
  Tanyakan pengantar rawat inap
\item
  Lakukan pemeriksaan anamnesis dan pemeriksaan fisik pasien
\item
  Konfirmasi instruksi dokter spesialis DPJP jika masih ada yang ragu atau kurang.
\item
  Lakukan pemeriksaan yang diinstruksikan DPJP
\item
  Berikan pengobatan yang diinstruksikan DPJP
\item
  Konsulkan hasil pemeriksaan penunjang ke DPJP
\item
  Lakukan konsul ke dr Spesialis lain jika diinstruksikan DPJP
\item
  Jika semua dr spesialis sudah acc tindakan atau semua instruksi telah dilakukan, konsul anestesi
\item
  Sampaikan hasil konsul dr anestesi ke DPJP
\end{enumerate}

\hypertarget{catatan-penting}{%
\chapter{Catatan Penting}\label{catatan-penting}}

\hypertarget{kronologi}{%
\section{Kronologi}\label{kronologi}}

Pasien trauma akan diminta membuat surat keterangan kronologi oleh FO, baik pasien BPJS maupun Umum.
Mohon disesuaikan penulisan MOI/Riwayat penyakit sekarang pada lembar Triage agar sependapat dengan lembar kronologi (bisa diiisi belakangan setelah kronologi dari FO selesai).
Kronologi ini penting terkait penjaminan biaya kesehatan.

\hypertarget{trauma-kepala}{%
\section{Trauma Kepala}\label{trauma-kepala}}

Pasien trauma dengan dugaan cidera kepala dan cidera lainnya, maka work up dan DPJP utamanya adalah Bedah Saraf. Jika terdapat cedera muskuloskeletal (masalah ortopedi) atau lainnya (masalah bedah lain) maka konsul setelah konsul DPJP bedah saraf. Untuk kasus CKR tanpa CT Scan, tanpa konsul bedah saraf, maka DPJP sesuai penyakit bedah Traumanya. Jika sudah CT scan, maka lebih baik dikonsulkan ke Bedah Saraf

Penulisan Diagnosis di lembar Triage juga diperhatikan agar diagnosa bedah saraf ditempatkan didepan. Contoh:

\begin{itemize}
\tightlist
\item
  CKS + EDH temporoparietal D + Fraktur clavicula + fraktur Humerus D.
\item
  CKB + Fraktur Depresi temporoparietal D + SDH tempral D + Dislokasi Glenohumeral Joint D
\end{itemize}

\hypertarget{kecelakaan}{%
\section{Kecelakaan}\label{kecelakaan}}

Pasien trauma akibat KLL wajib dibuatkan kronologi kejadian seperti aturan \protect\hyperlink{kronologi}{kronologi}
Kecelakaan dibagi menjadi 2:

\hypertarget{kecelakaan-lalu-lintas-kll}{%
\subsection{Kecelakaan Lalu Lintas (KLL)}\label{kecelakaan-lalu-lintas-kll}}

Kecelakaan lalu lintas ditangani secara medis sama seperti kasus bedah trauma dengan algoritma Primary Survery dan Secondary Survey.

\hypertarget{administratif}{%
\subsubsection{Administratif:}\label{administratif}}

Secara Administratif KLL memiliki beberapa ketentuan:

\begin{enumerate}
\def\labelenumi{\arabic{enumi}.}
\item
  KLL OC (\emph{out of control}) dan tunggal, tidak ditanggung oleh jasa Raharja. Pasien bisa ditanggung BPJS jika sudah mengurus surat keterangan polisi. Pengurusan surat keterangan polisi bisa memakan waktu, dan diberikan waktu 2 x 24 jam kerja. Sementara selama belum ada suket polisi, maka penjaminan pasien masih menjadi \textbf{UMUM}.
\item
  KLL dengan lawan bisa ditanggung jasa raharja. Penjaminan jasa raharja juga memerlukan laporan polisi dan pengurusan administrasi. Selama pengurusan itu status penjaminan biaya masih \textbf{UMUM}. Biaya Penjaminan jasa raharja untuk cedera berat adalah 20 juta. Jika pembiayaan melebihi tanggungan jasa raharja, maka penjaminan BPJS kesehatan akan berlaku. Oleh karena itu, kapasitas pembiayaan cukup besar.
\end{enumerate}

\hypertarget{kecelakaan-kerja-kk}{%
\subsection{Kecelakaan Kerja (KK)}\label{kecelakaan-kerja-kk}}

Kecelakaan kerja adalah cedera/kecelakaan yang terjadi pada saat proses bekerja baik kerja secara formal atau informal.
Kecelakaan kerja akan ditanggung dengan jaminan BPJS Ketenagakerjaan, tentu jika pekerja didaftarkan ke BPJS ketenagakerjaan oleh pemberi kerja. Jika pasien tidak memiliki BPJS ketenagakerjaan dan kecelakaan terjadi pada saat bekerja, konsulkan dulu ke TIM JKN untuk memastikan status penjaminan biayanya karena bisa tidak ditanggung BPJS Kesehatan.

\hypertarget{medikolegal}{%
\subsubsection{Medikolegal}\label{medikolegal}}

Aspek medikolegal pasien dilihat di bagian Forensik dan Medikolegal

\hypertarget{konsultasi}{%
\chapter{Konsultasi}\label{konsultasi}}

\hypertarget{pedoman-umum-konsultasi}{%
\section{Pedoman Umum Konsultasi}\label{pedoman-umum-konsultasi}}

\hypertarget{konsultasi-bidang-penyakit-dalam}{%
\section{Konsultasi Bidang Penyakit Dalam}\label{konsultasi-bidang-penyakit-dalam}}

\hypertarget{konsultasi-bidang-penyakit-paru}{%
\section{Konsultasi Bidang Penyakit Paru}\label{konsultasi-bidang-penyakit-paru}}

\hypertarget{konsultasi-bidang-penyakit-saraf}{%
\section{Konsultasi Bidang Penyakit Saraf}\label{konsultasi-bidang-penyakit-saraf}}

\hypertarget{konsultasi-bidang-penyakit-anak}{%
\section{Konsultasi Bidang Penyakit Anak}\label{konsultasi-bidang-penyakit-anak}}

\hypertarget{konsultasi-bidang-penyakit-ortopedi}{%
\section{Konsultasi Bidang Penyakit Ortopedi}\label{konsultasi-bidang-penyakit-ortopedi}}

\hypertarget{konsultasi-bidang-penyakit-bedah-umum-digestif-dan-onkologi}{%
\section{Konsultasi Bidang Penyakit Bedah Umum, Digestif dan Onkologi}\label{konsultasi-bidang-penyakit-bedah-umum-digestif-dan-onkologi}}

\hypertarget{konsultasi-bidang-penyakit-bedah-urologi}{%
\section{Konsultasi Bidang Penyakit Bedah Urologi}\label{konsultasi-bidang-penyakit-bedah-urologi}}

\hypertarget{konsultasi-bidang-anestesi-dan-terapi-intensif}{%
\section{Konsultasi Bidang Anestesi dan Terapi Intensif}\label{konsultasi-bidang-anestesi-dan-terapi-intensif}}

\hypertarget{konsultasi-bidang-telinga-hidung-dan-tenggorokan-tht}{%
\section{Konsultasi Bidang Telinga, Hidung dan Tenggorokan (THT)}\label{konsultasi-bidang-telinga-hidung-dan-tenggorokan-tht}}

\hypertarget{konsultasi-bidang-obstetri-dan-ginekologi}{%
\section{Konsultasi Bidang Obstetri dan Ginekologi}\label{konsultasi-bidang-obstetri-dan-ginekologi}}

\hypertarget{konsultasi-bidang-penyakit-mata}{%
\section{Konsultasi Bidang Penyakit Mata}\label{konsultasi-bidang-penyakit-mata}}

\hypertarget{konsultasi-bidang-penyakit-kulit-dan-kelain-dermatovenereologi}{%
\section{Konsultasi Bidang Penyakit Kulit dan Kelain (dermatovenereologi)}\label{konsultasi-bidang-penyakit-kulit-dan-kelain-dermatovenereologi}}

\hypertarget{penanganan-kasus-jiwa-psikiatri}{%
\section{Penanganan Kasus Jiwa (Psikiatri)}\label{penanganan-kasus-jiwa-psikiatri}}

\hypertarget{penanganan-kasus-forensik-dan-medikolegal}{%
\section{Penanganan Kasus Forensik dan Medikolegal}\label{penanganan-kasus-forensik-dan-medikolegal}}

\hypertarget{konsultasi-manajemen-dan-tim-jkn}{%
\section{Konsultasi Manajemen dan Tim JKN}\label{konsultasi-manajemen-dan-tim-jkn}}

\hypertarget{sharing-your-book}{%
\chapter{Sharing your book}\label{sharing-your-book}}

\hypertarget{publishing}{%
\section{Publishing}\label{publishing}}

HTML books can be published online, see: \url{https://bookdown.org/yihui/bookdown/publishing.html}

\hypertarget{pages}{%
\section{404 pages}\label{pages}}

By default, users will be directed to a 404 page if they try to access a webpage that cannot be found. If you'd like to customize your 404 page instead of using the default, you may add either a \texttt{\_404.Rmd} or \texttt{\_404.md} file to your project root and use code and/or Markdown syntax.

\hypertarget{metadata-for-sharing}{%
\section{Metadata for sharing}\label{metadata-for-sharing}}

Bookdown HTML books will provide HTML metadata for social sharing on platforms like Twitter, Facebook, and LinkedIn, using information you provide in the \texttt{index.Rmd} YAML. To setup, set the \texttt{url} for your book and the path to your \texttt{cover-image} file. Your book's \texttt{title} and \texttt{description} are also used.

This \texttt{bs4\_book} provides enhanced metadata for social sharing, so that each chapter shared will have a unique description, auto-generated based on the content.

Specify your book's source repository on GitHub as the \texttt{repo} in the \texttt{\_output.yml} file, which allows users to view each chapter's source file or suggest an edit. Read more about the features of this output format here:

\url{https://pkgs.rstudio.com/bookdown/reference/bs4_book.html}

Or use:

\begin{Shaded}
\begin{Highlighting}[]
\NormalTok{?bookdown}\SpecialCharTok{::}\NormalTok{bs4\_book}
\end{Highlighting}
\end{Shaded}


  \bibliography{book.bib,packages.bib}

\end{document}
