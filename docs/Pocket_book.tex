% Options for packages loaded elsewhere
\PassOptionsToPackage{unicode}{hyperref}
\PassOptionsToPackage{hyphens}{url}
%
\documentclass[
]{book}
\usepackage{lmodern}
\usepackage{amsmath}
\usepackage{ifxetex,ifluatex}
\ifnum 0\ifxetex 1\fi\ifluatex 1\fi=0 % if pdftex
  \usepackage[T1]{fontenc}
  \usepackage[utf8]{inputenc}
  \usepackage{textcomp} % provide euro and other symbols
  \usepackage{amssymb}
\else % if luatex or xetex
  \usepackage{unicode-math}
  \defaultfontfeatures{Scale=MatchLowercase}
  \defaultfontfeatures[\rmfamily]{Ligatures=TeX,Scale=1}
\fi
% Use upquote if available, for straight quotes in verbatim environments
\IfFileExists{upquote.sty}{\usepackage{upquote}}{}
\IfFileExists{microtype.sty}{% use microtype if available
  \usepackage[]{microtype}
  \UseMicrotypeSet[protrusion]{basicmath} % disable protrusion for tt fonts
}{}
\makeatletter
\@ifundefined{KOMAClassName}{% if non-KOMA class
  \IfFileExists{parskip.sty}{%
    \usepackage{parskip}
  }{% else
    \setlength{\parindent}{0pt}
    \setlength{\parskip}{6pt plus 2pt minus 1pt}}
}{% if KOMA class
  \KOMAoptions{parskip=half}}
\makeatother
\usepackage{xcolor}
\IfFileExists{xurl.sty}{\usepackage{xurl}}{} % add URL line breaks if available
\IfFileExists{bookmark.sty}{\usepackage{bookmark}}{\usepackage{hyperref}}
\hypersetup{
  pdftitle={The Gatekeeper's Notebook},
  pdfauthor={Kadek Adit Wiryadana},
  hidelinks,
  pdfcreator={LaTeX via pandoc}}
\urlstyle{same} % disable monospaced font for URLs
\usepackage{longtable,booktabs}
\usepackage{calc} % for calculating minipage widths
% Correct order of tables after \paragraph or \subparagraph
\usepackage{etoolbox}
\makeatletter
\patchcmd\longtable{\par}{\if@noskipsec\mbox{}\fi\par}{}{}
\makeatother
% Allow footnotes in longtable head/foot
\IfFileExists{footnotehyper.sty}{\usepackage{footnotehyper}}{\usepackage{footnote}}
\makesavenoteenv{longtable}
\usepackage{graphicx}
\makeatletter
\def\maxwidth{\ifdim\Gin@nat@width>\linewidth\linewidth\else\Gin@nat@width\fi}
\def\maxheight{\ifdim\Gin@nat@height>\textheight\textheight\else\Gin@nat@height\fi}
\makeatother
% Scale images if necessary, so that they will not overflow the page
% margins by default, and it is still possible to overwrite the defaults
% using explicit options in \includegraphics[width, height, ...]{}
\setkeys{Gin}{width=\maxwidth,height=\maxheight,keepaspectratio}
% Set default figure placement to htbp
\makeatletter
\def\fps@figure{htbp}
\makeatother
\setlength{\emergencystretch}{3em} % prevent overfull lines
\providecommand{\tightlist}{%
  \setlength{\itemsep}{0pt}\setlength{\parskip}{0pt}}
\setcounter{secnumdepth}{5}
\usepackage{booktabs}
\ifluatex
  \usepackage{selnolig}  % disable illegal ligatures
\fi
\usepackage[]{natbib}
\bibliographystyle{apalike}

\title{The Gatekeeper's Notebook}
\author{Kadek Adit Wiryadana}
\date{2021-12-02}

\begin{document}
\maketitle

{
\setcounter{tocdepth}{1}
\tableofcontents
}
\hypertarget{sepatah-kata}{%
\chapter{Sepatah Kata}\label{sepatah-kata}}

Buku ini disusun untuk dokter Jaga Unit Gawat Darurat. Buku berisikan panduan umum berdasarkan pengalaman penulis beserta hasil diskusi dengan teman sejawat dan konsultasi dengan dokter Spesialis dan Managemen.

\hypertarget{panduan}{%
\section{Panduan}\label{panduan}}

Buku ini hanya berisikan panduan umum. Buku tidak disusun dengan tujuan menjadi panduan komprehensif ataupun Cheatsheet. Tetap sesuaikan dengan kondisi dan selalu dengarkan kata hati.

\hypertarget{narahubung}{%
\section{Narahubung}\label{narahubung}}

Buku disusun oleh dr. Kadek Adit Wiryadana, pertanyaan dan saran dapat disampaikan pada \href{mailto:ka.wiryadana@gmail.com}{\nolinkurl{ka.wiryadana@gmail.com}}

\hypertarget{smf-dokter-umum}{%
\chapter{SMF Dokter Umum}\label{smf-dokter-umum}}

Para Dokter Umum di RS bernaung dibawah satuan medik fungsional (SMF) tersendiri yaitu SMF Dokter Umum

\hypertarget{struktur-dan-anggota-smf-dokter-umum}{%
\section{Struktur dan Anggota SMF Dokter Umum}\label{struktur-dan-anggota-smf-dokter-umum}}

\hypertarget{koordinator-2022}{%
\subsection{Koordinator (2022)}\label{koordinator-2022}}

TBD

\hypertarget{pelaksana-tugas-sementara-koordinator-dokter-umum-2021}{%
\subsection{Pelaksana Tugas Sementara Koordinator Dokter Umum (2021)}\label{pelaksana-tugas-sementara-koordinator-dokter-umum-2021}}

\begin{itemize}
\tightlist
\item
  dr. Kadek Aditya Dharmayoga (\textbf{Fulltime})
\end{itemize}

\hypertarget{anggota}{%
\subsection{Anggota}\label{anggota}}

\begin{itemize}
\tightlist
\item
  dr. Putu Eka Putra Adnyana
\item
  dr. Dwi Meviyanti
\item
  dr. M. Azman Pasha
\item
  dr. Iska (\textbf{Fulltime})
\item
  dr. I.B Anom Dharma Putra
\item
  dr. Gede Dewagama
\item
  dr. Michael Oliver Wijaya
\item
  dr. Kadek Dwi Wira Sanjaya
\item
  dr. Ngurah Agung Reza Satria Nugraha Putra (\textbf{Fulltime})
\item
  dr. Kadek Adit Wiryadana (\textbf{Fulltime})
\item
  dr. Agustin J Nanda De niro (\textbf{Fulltime})
\item
  dr. Made Vidyasti Laksita Wijaya (\textbf{Fulltime})
\item
  dr. I Gusti Agung Anggia Noverina (\textbf{Hemodialisis})
\item
  dr. Gracia Dewitacita Tanaya
\end{itemize}

\hypertarget{peraturan-kerja}{%
\chapter{Peraturan Kerja}\label{peraturan-kerja}}

Hal-hal yang perlu dipatuhi saat bekerja

\hypertarget{jam-kerja}{%
\section{Jam Kerja}\label{jam-kerja}}

Jam kerja dibagi menjadi tiga shift.

\begin{itemize}
\item
  Pagi (08.00 - 14.00 Wita)
\item
  Sore (14.00 - 20.00 Wita)
\item
  Malam (20.00 - 08.00 Wita)
\end{itemize}

\hypertarget{poin-kerja}{%
\section{Poin Kerja}\label{poin-kerja}}

\hypertarget{khusus-fultime}{%
\subsection*{khusus Fultime}\label{khusus-fultime}}
\addcontentsline{toc}{subsection}{khusus Fultime}

Sesuai dengan kontrak, jaga dihitung dengan poin dengan rincian:

\begin{itemize}
\item
  Jaga Pagi \textasciitilde{} 1 poin
\item
  Jaga Sore \textasciitilde{} 1 Poin
\item
  Jaga Malam \textasciitilde{} 2 Poin
\end{itemize}

Dokter kontrak dalam sebulan minimal mendapatkan 24 poin.

\hypertarget{poin-kerja-part-time}{%
\subsection*{Poin Kerja Part-time}\label{poin-kerja-part-time}}
\addcontentsline{toc}{subsection}{Poin Kerja Part-time}

Dokter Part-time diberikan kesempatan untuk memilih waktu jaga, jika ada shift dimana tidak bisa diisi oleh dokter fulltime.

\hypertarget{pakaian-kerja}{%
\section{Pakaian Kerja}\label{pakaian-kerja}}

\begin{itemize}
\tightlist
\item
  Saat datang ke Rumah Sakit diharapkan berpakaian Rapi.
\item
  Saat bekerja menggunakan set baju jaga yang telah disediakan (baju, celana, headcap, alas kaki), kecuali ukuran baju jaga tidak cukup
\item
  Menggunakan masker N95 atau KN95
\item
  Menggunakan baju operasi (gown)
\end{itemize}

\hypertarget{cross}{%
\chapter{Alur Kerja}\label{cross}}

\hypertarget{pasien-baru-ugd}{%
\section{Pasien Baru UGD}\label{pasien-baru-ugd}}

\begin{enumerate}
\def\labelenumi{\arabic{enumi}.}
\tightlist
\item
  Pasien datang ke UGD
\item
  Tanyakan keluhan utama
\item
  Screening TTV dan SpO2
\item
  Lakukan pemeriksaan anamnesis dan pemeriksaan fisik
\item
  KIE pemeriksaan penunjang untuk penegakan diagnosis dan KIE waktu tunggu
\item
  Berikan pengobatan awal dan lakukan pemeriksaan penunjang
\item
  Konsul ke dr Spesialis terkait jika data cukup
\item
  Kie pasien dan keluarga terkait diagnosis dan pengobatan yang akan diberikan
\item
  Berikan pengobatan.
\end{enumerate}

\hypertarget{pasien-rujukan-poli}{%
\section{Pasien rujukan Poli}\label{pasien-rujukan-poli}}

\begin{enumerate}
\def\labelenumi{\arabic{enumi}.}
\tightlist
\item
  Pasien diantar perawat poli ke UGD.
\item
  Operan dan baca pengantar rawat inap dari dr Spesialis.
\item
  Lakukan pemeriksaan anamnesis dan pemeriksaan fisik pasien
\item
  Konfirmasi instruksi dokter spesialis jika masih ada yang ragu atau kurang.
\item
  Lakukan pemeriksaan yang diinstruksikan dr Spesialis
\item
  Berikan pengobatan yang diinstruksikan dr Spesialis
\item
  Konsulkan hasil pemeriksaan penunjang ke dr Spesialis
\end{enumerate}

\hypertarget{pasien-pre-op}{%
\section{Pasien Pre-OP}\label{pasien-pre-op}}

\begin{enumerate}
\def\labelenumi{\arabic{enumi}.}
\tightlist
\item
  Pasien diantar keluarga ke UGD
\item
  Tanyakan keluhan utama, jika dikatakan rencana operasi.
\item
  Tanyakan pengantar rawat inap
\item
  Lakukan pemeriksaan anamnesis dan pemeriksaan fisik pasien
\item
  Konfirmasi instruksi dokter spesialis DPJP jika masih ada yang ragu atau kurang.
\item
  Lakukan pemeriksaan yang diinstruksikan DPJP
\item
  Berikan pengobatan yang diinstruksikan DPJP
\item
  Konsulkan hasil pemeriksaan penunjang ke DPJP
\item
  Lakukan konsul ke dr Spesialis lain jika diinstruksikan DPJP
\item
  Jika semua dr spesialis sudah acc tindakan atau semua instruksi telah dilakukan, konsul anestesi
\item
  Sampaikan hasil konsul dr anestesi ke DPJP
\end{enumerate}

\hypertarget{catatan-penting}{%
\chapter{Catatan Penting}\label{catatan-penting}}

\hypertarget{screening-covid}{%
\section{Screening COVID}\label{screening-covid}}

Screening covid wajib meliputi screening EWS serta rapid antigen. Xray thorax bukan pemeriksaan screening rutin, lakukan jika ada indikasi.
Indikasi meliputi: tanda klinis ISPA atau pneumonia, kemungkinan kardiomegali (HHD, konsul jantung)

\hypertarget{kronologi}{%
\section{Kronologi}\label{kronologi}}

Pasien trauma akan diminta membuat surat keterangan kronologi oleh FO, baik pasien BPJS maupun Umum.
Mohon disesuaikan penulisan MOI/Riwayat penyakit sekarang pada lembar Triage agar sependapat dengan lembar kronologi (bisa diiisi belakangan setelah kronologi dari FO selesai).
Kronologi ini penting terkait penjaminan biaya kesehatan.

\hypertarget{trauma-kepala}{%
\section{Trauma Kepala}\label{trauma-kepala}}

Pasien trauma dengan dugaan cidera kepala dan cidera lainnya, maka work up dan DPJP utamanya adalah Bedah Saraf. Jika terdapat cedera muskuloskeletal (masalah ortopedi) atau lainnya (masalah bedah lain) maka konsul setelah konsul DPJP bedah saraf. Untuk kasus CKR tanpa CT Scan, tanpa konsul bedah saraf, maka DPJP sesuai penyakit bedah Traumanya. Jika sudah CT scan, maka lebih baik dikonsulkan ke Bedah Saraf

Penulisan Diagnosis di lembar Triage juga diperhatikan agar diagnosa bedah saraf ditempatkan didepan. Contoh:

\begin{itemize}
\tightlist
\item
  CKS + EDH temporoparietal D + Fraktur clavicula + fraktur Humerus D.
\item
  CKB + Fraktur Depresi temporoparietal D + SDH tempral D + Dislokasi Glenohumeral Joint D
\end{itemize}

\hypertarget{kecelakaan}{%
\section{Kecelakaan}\label{kecelakaan}}

Pasien trauma akibat KLL wajib dibuatkan kronologi kejadian seperti aturan \protect\hyperlink{kronologi}{kronologi}
Kecelakaan dibagi menjadi 2:

\hypertarget{kecelakaan-lalu-lintas-kll}{%
\subsection{Kecelakaan Lalu Lintas (KLL)}\label{kecelakaan-lalu-lintas-kll}}

Kecelakaan lalu lintas ditangani secara medis sama seperti kasus bedah trauma dengan algoritma Primary Survery dan Secondary Survey.

\hypertarget{administratif}{%
\subsubsection{Administratif:}\label{administratif}}

Secara Administratif KLL memiliki beberapa ketentuan:

\begin{enumerate}
\def\labelenumi{\arabic{enumi}.}
\item
  KLL OC (\emph{out of control}) dan tunggal, tidak ditanggung oleh jasa Raharja. Pasien bisa ditanggung BPJS jika sudah mengurus surat keterangan polisi. Pengurusan surat keterangan polisi bisa memakan waktu, dan diberikan waktu 2 x 24 jam kerja. Sementara selama belum ada suket polisi, maka penjaminan pasien masih menjadi \textbf{UMUM}.
\item
  KLL dengan lawan bisa ditanggung jasa raharja. Penjaminan jasa raharja juga memerlukan laporan polisi dan pengurusan administrasi. Selama pengurusan itu status penjaminan biaya masih \textbf{UMUM}. Biaya Penjaminan jasa raharja untuk cedera berat adalah 20 juta. Jika pembiayaan melebihi tanggungan jasa raharja, maka penjaminan BPJS kesehatan akan berlaku. Oleh karena itu, kapasitas pembiayaan cukup besar.
\end{enumerate}

\hypertarget{kecelakaan-kerja-kk}{%
\subsection{Kecelakaan Kerja (KK)}\label{kecelakaan-kerja-kk}}

Kecelakaan kerja adalah cedera/kecelakaan yang terjadi pada saat proses bekerja baik kerja secara formal atau informal.
Kecelakaan kerja akan ditanggung dengan jaminan BPJS Ketenagakerjaan, tentu jika pekerja didaftarkan ke BPJS ketenagakerjaan oleh pemberi kerja. Jika pasien tidak memiliki BPJS ketenagakerjaan dan kecelakaan terjadi pada saat bekerja, konsulkan dulu ke TIM JKN untuk memastikan status penjaminan biayanya karena bisa tidak ditanggung BPJS Kesehatan.

\hypertarget{medikolegal}{%
\subsubsection{Medikolegal}\label{medikolegal}}

Aspek medikolegal pasien dilihat di bagian Forensik dan Medikolegal

\hypertarget{konsultasi}{%
\chapter{Konsultasi}\label{konsultasi}}

Konsultasi dilakukan jika pasien sesuai dengan expertise medis memerlukan rawat inap. Pada kasus dan kondisi tertentu, pasien rawat jalan UGD dapat dikonsultasikan jika membutuhkan penanganan medis spesialistik rawat jalan atau jika pasien sudah biasa kontrol dengan dokter spesialisnya atau melakukan pemeriksaan Lab di rumah sakit atas perintah dokter spesialis.

\hypertarget{pedoman-umum-konsultasi}{%
\section{Pedoman Umum Konsultasi}\label{pedoman-umum-konsultasi}}

\hypertarget{konsultasi-via-telpon}{%
\subsection{Konsultasi via telpon}\label{konsultasi-via-telpon}}

Konsul via telepon dapat dilakukan via panggilan seluler GSM, panggilan Whatsapp atau pesan whatsapp.
Konsul via telepon harus diikuti dengan TULBAKON, terutama pada panggilan telepon.

\begin{itemize}
\tightlist
\item
  TUL: Tulis dikertas catatan
\item
  BA: Baca ulang ke konsultan
\item
  KON: Konfirmasi apakah sudah semua instruksi sudah dibacakan atau masih ada yang kurang
\end{itemize}

Tulbakon diikuti cap yang diisi nama dokter jaga, tanggal serta jam konsul dan ditandatangani.
Cap tulbakon diisi pada bagian kosong lembar TRIAGE atau CPPT (jika pasien status rawat inap)

\hypertarget{konsultasi-dan-visite-spesialis-di-ugd}{%
\subsection{Konsultasi dan Visite Spesialis di UGD}\label{konsultasi-dan-visite-spesialis-di-ugd}}

Konsultasi dan jika disertai visite dokter spesialis, maka dokter spesialis menulis instruksi pada lembar CPPT. Hasil instruksi dr Spesialis tetap disalin pada lembar TRIAGE secara singkat (tanpa cap tulbakon).

\hypertarget{menghubungi-dokter-spesialis}{%
\subsection{Menghubungi Dokter Spesialis}\label{menghubungi-dokter-spesialis}}

\begin{itemize}
\tightlist
\item
  Hubungi awal (via WA atau telepon) dokter Spesialis
\item
  Jika dalam 15 menit pesan wa awal belum dibalas atau telpon pertama tidak diangkat, lakukan telpon ulang 1.
\item
  jika telepon ulang 1 belum diangkat, hubungi kembali DPJP 15 menit kemudian (telepon ulang 2)
\item
  Jika telepon ulang 2 belum diangkat, hubungi kembali DPJP 15 menit kemudian (telepon ulang 3)
\item
  Jika telepon ulang 3 belum diangkat, maka dokter jaga dapat mengalihkan DPJP.
\end{itemize}

\hypertarget{dokter-spesialis-yang-bertanggung-jawab-atas-pasien}{%
\subsection{Dokter Spesialis yang bertanggung jawab atas pasien}\label{dokter-spesialis-yang-bertanggung-jawab-atas-pasien}}

\begin{enumerate}
\def\labelenumi{\arabic{enumi}.}
\tightlist
\item
  DPJP pasien baru sesuai dengan jadwal dokter spesialis jaga.
\item
  DPJP pasien yang sebelumnya sudah pernah dirawat oleh seorang dokter spesialis, dan datang dengan keluhan yang sesuai dengan ranah/kewenangan medis dr Spesialis tersebut, maka dikonsulkan ke dokter spesialis yang bersangkutan.
\item
  DPJP dapat dialihkan karena suatu alasan (Dokter spesialis berhalangan, mengganti jadwal jaga, Cuti) jika dilakukan permintaan oleh dokter spesialis sebelumnya dan disetujui oleh dokter spesialis yang akan menjadi DPJP.
\item
  Pasien Umum dapat memilih dokter Spesialis yang akan merawatnya.
\end{enumerate}

\hypertarget{prinsip-format-pelaporan}{%
\subsection{Prinsip Format Pelaporan}\label{prinsip-format-pelaporan}}

\begin{enumerate}
\def\labelenumi{\arabic{enumi}.}
\tightlist
\item
  Salam dan Perkenalan singkat
\item
  Identitas Pasien (Nama, Usia, Jenis Kelamin)
\item
  Subjektif: berisi keluhan dan gejala pasien, Riwayat Penyakit Dahulu, Riwayat Pengobatan
\item
  Objektif: Tanda Vital, Pemeriksaan Generalis, Status Lokalis
\item
  Assesment, diganosis kerja
\item
  Planning, tatalaksana yang telah diberikan
\item
  Penutup
\item
  Lampiran hasil pemeriksaan penunjang
\end{enumerate}

\hypertarget{contoh-format-umum}{%
\subsection{Contoh Format Umum}\label{contoh-format-umum}}

Selamat malam dr Yogi Sp,PD,

Mohon maaf menggangu, saya dr W, dr Jaga RS Balimed.

Ijin mengkonsulkan pasien baru di IGD

Made Batu/60/L

S: Penurunan Kesadaran

Pasien dikatakan tidak nyambung diajak berbicara sejak 6 jam SMRS. Sejak 2 jam SMRS pasien tampak tertidur dan sulit dibangungkan. Pasien sebelumnya mengeluh lemas diseluruh tubuh, keringan dingin, nyeri kepala dan tidak mau makan. Keluhan mual dan muntah disangkal. Kelemahan separuh tubuh kanan/kiri disangkal.

Keluhan batuk, sejak, demam disangkal. Vaksin COVID (2x)

RPD:
Diabetes tipe 2 (+) sejak 10 tahun tidak teratur pengobatan. HT, Jantung, stroke, ginjal disangkal

RPO:
Glibenklamid 2 x 1 tablet, Metformin 3x1 tablet

O:
* GCS E2V3M3

\begin{itemize}
\item
  TD: 130/70
\item
  HR: 126x/min
\item
  RR: 22x/min
\item
  Temp: 37,4
\item
  SpO2: 94\%
\end{itemize}

Mata: anemis -/-, Ikterik -/-, r.pupil +/+ isokor 3/3 mm

Leher: pkgb (-), JVP: dbn

Thorax: simetris saat statis dan dinamis, retraksi (-)

COR: S1S2 Tunggal, Reguler, Murmur (-), Gallop (-)

Pulmo: Ves +/+, rhonki -/-, wheezing -/-

Abdomen: Distensi (-), BU (+) meningkat, timpani, nyeri tekan sulit dievaluasi

Extremitas: hangat +/+, edema -/-

GDS Stick 40 g/dl

A:

DOC ec susp. hipoglikemia + DM tipe 2

P:

\begin{itemize}
\tightlist
\item
  Nasal Canul 2 lpm
\item
  Loading D40\% 2 flakon lanjut IVFD D10\% 20 tpm
\item
  Ranitidin 50 mg (IV)
\item
  Observasi, dan GDA evaluasi 15-20 menit pasca pemberian D40
\item
  Pemeriksaan DL, SGOT SGPT, BUN SC, SE, Rapid Antigen, EKG
\end{itemize}

Ijin melampirkan hasil pemeriksaan penunjang dokter.
Mohon adviz dan terapinya dokter

\begin{center}\rule{0.5\linewidth}{0.5pt}\end{center}

\hypertarget{konsultasi-bidang-penyakit-dalam}{%
\section{Konsultasi Bidang Penyakit Dalam}\label{konsultasi-bidang-penyakit-dalam}}

\textbf{Dokter Spesialis}

\begin{enumerate}
\def\labelenumi{\arabic{enumi}.}
\tightlist
\item
  dr. Andriyasa
\item
  dr. Ratih
\item
  dr. Yogi
\end{enumerate}

\hypertarget{pedoman-umum-interna}{%
\subsection{Pedoman Umum Interna}\label{pedoman-umum-interna}}

Mengikuti Contoh Peduman umum

\hypertarget{ketentuan-khusus-dokter-spesialis}{%
\subsection{Ketentuan Khusus Dokter Spesialis}\label{ketentuan-khusus-dokter-spesialis}}

\textbf{dr. Andriyasa}

\begin{enumerate}
\def\labelenumi{\arabic{enumi}.}
\tightlist
\item
  Konsul dapat dilakukan via pesan WA atau telepon. Respon lebih cepat dan diharapkan via telepon.
\item
  Menurut dr Azman, beliau tidak keberatan jika kita lakukan terapi dulu sesuai kemampuan kita dan pengalaman konsul dengan beliau sebelumnya.
  Konsul dengan beliau harus mengerti problem list dari pasien dan hasil pemeriksaan yang dalam batas normal tidak perlu disampaikan.
\end{enumerate}

\textbf{dr. Ratih}

\begin{enumerate}
\def\labelenumi{\arabic{enumi}.}
\tightlist
\item
  Konsul diharapkan via pesan WA.
\end{enumerate}

\textbf{dr. Yogi}

\begin{enumerate}
\def\labelenumi{\arabic{enumi}.}
\tightlist
\item
  Konsul dapat dilakukan via pesan WA atau telepon.
\item
  dr. Yogi juga merawat pasien COVID, jika pasien dengan diagnosis COVID-19 dan DPJP beliau, tidak perlu dikonsulkan paru lagi.
\end{enumerate}

\begin{center}\rule{0.5\linewidth}{0.5pt}\end{center}

\hypertarget{konsultasi-bidang-penyakit-paru}{%
\section{Konsultasi Bidang Penyakit Paru}\label{konsultasi-bidang-penyakit-paru}}

\textbf{Dokter Spesialis}

\begin{enumerate}
\def\labelenumi{\arabic{enumi}.}
\tightlist
\item
  dr. Adisty
\item
  dr. Agustya
\end{enumerate}

\hypertarget{pedoman-tambahan-paru}{%
\subsection{Pedoman Tambahan Paru}\label{pedoman-tambahan-paru}}

Bisa dijabarkan lebih pada pemeriksaan thorax

Thorax:

Inspeksi: simetris saat statis dan dinamis, retraksi (-)

Palpasi: Fremitus taktil N/N, nyeri tekan (-)

Perkusi:Sonor

\begin{verbatim}
       +/+
         
       +/+
             
       +/+
             
\end{verbatim}

AUskultasi: Vesikuler, rhonki, wheezing

\begin{verbatim}
      +/+          -/-           -/-
                    
      +/+          -/-           -/-
      
      +/+          -/-           -/-
\end{verbatim}

\begin{center}\rule{0.5\linewidth}{0.5pt}\end{center}

\hypertarget{konsultasi-bidang-penyakit-saraf}{%
\section{Konsultasi Bidang Penyakit Saraf}\label{konsultasi-bidang-penyakit-saraf}}

\textbf{Dokter Spesialis}

\begin{enumerate}
\def\labelenumi{\arabic{enumi}.}
\tightlist
\item
  dr. Lina
\item
  dr. Phala
\end{enumerate}

\hypertarget{pedoman-umum}{%
\subsection{Pedoman Umum}\label{pedoman-umum}}

\hypertarget{contoh-format}{%
\subsection{Contoh Format}\label{contoh-format}}

\hypertarget{ketentuan-khusus-dokter-spesialis-1}{%
\subsection{Ketentuan Khusus Dokter Spesialis}\label{ketentuan-khusus-dokter-spesialis-1}}

\textbf{dr. Phala}

\begin{enumerate}
\def\labelenumi{\arabic{enumi}.}
\item
  Lakukan pemeriksaan sesuai instruksi beliau. Jika ragu, bisa konfirmasi atau konsul dulu.
\item
  Ikuti instruksi diagnosis dan terapi IGD sesuai instruksi beliau.
\end{enumerate}

\begin{center}\rule{0.5\linewidth}{0.5pt}\end{center}

\hypertarget{konsultasi-bidang-penyakit-anak}{%
\section{Konsultasi Bidang Penyakit Anak}\label{konsultasi-bidang-penyakit-anak}}

\textbf{Dokter Spesialis}

\begin{enumerate}
\def\labelenumi{\arabic{enumi}.}
\tightlist
\item
  dr. Suciawan
\item
  dr. Lucky
\item
  dr. Wahyuni
\end{enumerate}

\hypertarget{pedoman-umum-1}{%
\subsection{Pedoman Umum}\label{pedoman-umum-1}}

\hypertarget{contoh-format-1}{%
\subsection{Contoh Format}\label{contoh-format-1}}

\begin{center}\rule{0.5\linewidth}{0.5pt}\end{center}

\hypertarget{konsultasi-bidang-penyakit-ortopedi}{%
\section{Konsultasi Bidang Penyakit Ortopedi}\label{konsultasi-bidang-penyakit-ortopedi}}

\textbf{Dokter Spesialis}

\begin{enumerate}
\def\labelenumi{\arabic{enumi}.}
\tightlist
\item
  dr. Kusuma
\item
  dr. Sumadi
\end{enumerate}

\hypertarget{pedoman-umum-2}{%
\subsection{Pedoman Umum}\label{pedoman-umum-2}}

\hypertarget{contoh-format-2}{%
\subsection{Contoh Format}\label{contoh-format-2}}

\begin{center}\rule{0.5\linewidth}{0.5pt}\end{center}

\hypertarget{konsultasi-bidang-penyakit-bedah-umum-digestif-dan-onkologi}{%
\section{Konsultasi Bidang Penyakit Bedah Umum, Digestif dan Onkologi}\label{konsultasi-bidang-penyakit-bedah-umum-digestif-dan-onkologi}}

\textbf{Dokter Spesialis}

\hypertarget{bedah-umum}{%
\subsubsection{Bedah Umum}\label{bedah-umum}}

\begin{enumerate}
\def\labelenumi{\arabic{enumi}.}
\tightlist
\item
  dr. Oka
\item
  dr. Agus
\end{enumerate}

\hypertarget{bedah-digestif}{%
\subsubsection{Bedah Digestif}\label{bedah-digestif}}

\begin{enumerate}
\def\labelenumi{\arabic{enumi}.}
\tightlist
\item
  dr. Cok
\end{enumerate}

\hypertarget{bedah-onkologi}{%
\subsubsection{Bedah Onkologi}\label{bedah-onkologi}}

\begin{enumerate}
\def\labelenumi{\arabic{enumi}.}
\tightlist
\item
  dr. Suparna
\end{enumerate}

\hypertarget{pedoman-umum-3}{%
\subsection{Pedoman Umum}\label{pedoman-umum-3}}

\hypertarget{contoh-format-3}{%
\subsection{Contoh Format}\label{contoh-format-3}}

\hypertarget{ketentuan-khusus-dokter-spesialis-2}{%
\subsection{Ketentuan Khusus Dokter Spesialis}\label{ketentuan-khusus-dokter-spesialis-2}}

\hypertarget{dr.-oka-sp.b}{%
\subsubsection{dr. Oka Sp.B}\label{dr.-oka-sp.b}}

\begin{enumerate}
\def\labelenumi{\arabic{enumi}.}
\tightlist
\item
  Tumor soft tissue minimal ukuran 4x4 cm.
\item
  \textbf{Pre OP}
\end{enumerate}

\begin{itemize}
\tightlist
\item
  Pasien dibawah 40 tahun: DL, BT,CT. Rontgen Thorax dilakukan jika ada indikasi medis.
\item
  Pasien diatas 40 tahun: Lab Lengkap (DL, BT,CT, Bun, Sc, SGOT, SGPT, GDS) dan Thorax. EKG jika ada indikasi
\end{itemize}

\begin{enumerate}
\def\labelenumi{\arabic{enumi}.}
\setcounter{enumi}{2}
\tightlist
\item
  Jika pasien Pre OP memerlukan anestesi general dan diminta telah menjalani PCR sebelum dilakukan anestesi, maka pasien saat datang pagi kemudian dilakukan PCR, operasi dijadwalkan jam 15.00 setelah hasil PCR keluar.
\end{enumerate}

\begin{center}\rule{0.5\linewidth}{0.5pt}\end{center}

\hypertarget{konsultasi-bidang-penyakit-bedah-urologi}{%
\section{Konsultasi Bidang Penyakit Bedah Urologi}\label{konsultasi-bidang-penyakit-bedah-urologi}}

\textbf{Dokter Spesialis}

\begin{enumerate}
\def\labelenumi{\arabic{enumi}.}
\tightlist
\item
  dr. Reza
\end{enumerate}

\hypertarget{pedoman-umum-4}{%
\subsection{Pedoman Umum}\label{pedoman-umum-4}}

\hypertarget{contoh-format-4}{%
\subsection{Contoh Format}\label{contoh-format-4}}

\begin{center}\rule{0.5\linewidth}{0.5pt}\end{center}

\hypertarget{konsultasi-bidang-anestesi-dan-terapi-intensif}{%
\section{Konsultasi Bidang Anestesi dan Terapi Intensif}\label{konsultasi-bidang-anestesi-dan-terapi-intensif}}

\textbf{Dokter Spesialis}

\begin{enumerate}
\def\labelenumi{\arabic{enumi}.}
\tightlist
\item
  dr. Agus
\item
  dr. Suryawan
\item
  dr. Nova
\item
  dr. Darma
\item
  dr. Ari
\end{enumerate}

\hypertarget{pedoman-umum-5}{%
\subsection{Pedoman Umum}\label{pedoman-umum-5}}

\hypertarget{contoh-format-5}{%
\subsection{Contoh Format}\label{contoh-format-5}}

\hypertarget{ketentuan-khusus-dokter-spesialis-3}{%
\subsection{Ketentuan Khusus Dokter Spesialis}\label{ketentuan-khusus-dokter-spesialis-3}}

\begin{center}\rule{0.5\linewidth}{0.5pt}\end{center}

\hypertarget{konsultasi-bidang-telinga-hidung-dan-tenggorokan-tht}{%
\section{Konsultasi Bidang Telinga, Hidung dan Tenggorokan (THT)}\label{konsultasi-bidang-telinga-hidung-dan-tenggorokan-tht}}

\textbf{Dokter Spesialis}

\begin{enumerate}
\def\labelenumi{\arabic{enumi}.}
\tightlist
\item
  dr. Dwi Agustin
\end{enumerate}

\hypertarget{pedoman-umum-6}{%
\subsection{Pedoman Umum}\label{pedoman-umum-6}}

\hypertarget{contoh-format-6}{%
\subsection{Contoh Format}\label{contoh-format-6}}

\begin{center}\rule{0.5\linewidth}{0.5pt}\end{center}

\hypertarget{konsultasi-bidang-obstetri-dan-ginekologi}{%
\section{Konsultasi Bidang Obstetri dan Ginekologi}\label{konsultasi-bidang-obstetri-dan-ginekologi}}

\textbf{Dokter Spesialis}

\begin{enumerate}
\def\labelenumi{\arabic{enumi}.}
\tightlist
\item
  dr. Sudarsana
\item
  dr. Swastika
\item
  dr. Setya Budhidarma
\item
  dr. Aditya
\item
  dr. Rai
\item
  dr.
\end{enumerate}

\hypertarget{pedoman-umum-7}{%
\subsection{Pedoman Umum}\label{pedoman-umum-7}}

\hypertarget{contoh-format-7}{%
\subsection{Contoh Format}\label{contoh-format-7}}

\begin{center}\rule{0.5\linewidth}{0.5pt}\end{center}

\hypertarget{konsultasi-bidang-penyakit-mata}{%
\section{Konsultasi Bidang Penyakit Mata}\label{konsultasi-bidang-penyakit-mata}}

\textbf{Dokter Spesialis}

dr. Ayu Thea

\hypertarget{pedoman-umum-8}{%
\subsection{Pedoman Umum}\label{pedoman-umum-8}}

\hypertarget{contoh-format-8}{%
\subsection{Contoh Format}\label{contoh-format-8}}

\begin{center}\rule{0.5\linewidth}{0.5pt}\end{center}

\hypertarget{konsultasi-bidang-penyakit-kulit-dan-kelamin-dermatovenereologi}{%
\section{Konsultasi Bidang Penyakit Kulit dan Kelamin (dermatovenereologi)}\label{konsultasi-bidang-penyakit-kulit-dan-kelamin-dermatovenereologi}}

\textbf{Dokter Spesialis}

dr. Ketut Suteja Wibawa Sp.KK, M.Kes

\hypertarget{pedoman-umum-9}{%
\subsection{Pedoman Umum}\label{pedoman-umum-9}}

\hypertarget{contoh-format-9}{%
\subsection{Contoh Format}\label{contoh-format-9}}

\begin{center}\rule{0.5\linewidth}{0.5pt}\end{center}

\hypertarget{penanganan-kasus-jiwa-psikiatri}{%
\section{Penanganan Kasus Jiwa (Psikiatri)}\label{penanganan-kasus-jiwa-psikiatri}}

\begin{center}\rule{0.5\linewidth}{0.5pt}\end{center}

\hypertarget{penanganan-kasus-forensik-dan-medikolegal}{%
\section{Penanganan Kasus Forensik dan Medikolegal}\label{penanganan-kasus-forensik-dan-medikolegal}}

\textbf{Dokter Spesialis}

dr. Klarisa Sp.FM

\hypertarget{pedoman-umum-10}{%
\subsection{Pedoman Umum}\label{pedoman-umum-10}}

\hypertarget{contoh-format-10}{%
\subsection{Contoh Format}\label{contoh-format-10}}

\begin{center}\rule{0.5\linewidth}{0.5pt}\end{center}

\hypertarget{konsultasi-pembacaan-radiologi-cito}{%
\section{Konsultasi Pembacaan Radiologi CITO}\label{konsultasi-pembacaan-radiologi-cito}}

\textbf{Dokter Spesialis}

dr. Winda Sp.Rad

\hypertarget{pedoman-umum-11}{%
\subsection{Pedoman Umum}\label{pedoman-umum-11}}

\hypertarget{contoh-format-11}{%
\subsection{Contoh Format}\label{contoh-format-11}}

\begin{center}\rule{0.5\linewidth}{0.5pt}\end{center}

\hypertarget{konsultasi-manajemen-dan-tim-jkn}{%
\section{Konsultasi Manajemen dan Tim JKN}\label{konsultasi-manajemen-dan-tim-jkn}}

\begin{enumerate}
\def\labelenumi{\arabic{enumi}.}
\tightlist
\item
  dr. Adi Suryana (Kepala Divisi Pelayanan dan Medis)
\end{enumerate}

Konsul dr Adi jika dokter jaga belum mengerti alur administratif pasien/kasus khusus. Jika tidak emergency, bisa melalui FO ke tim JKN.

\begin{enumerate}
\def\labelenumi{\arabic{enumi}.}
\setcounter{enumi}{1}
\tightlist
\item
  Dewa Ryan (PIC BPJS)
\end{enumerate}

Konsul bapak Dewa Ryan jika terdapat alur administratifn BPJS (biasanya melalui FO atau perawat)

\begin{enumerate}
\def\labelenumi{\arabic{enumi}.}
\setcounter{enumi}{2}
\tightlist
\item
  Kudo (IT)
\end{enumerate}

Konsultasi terkait SIMARS dan IT bagian pelayanan lainnya (biasanya dilakukan perawat.

\hypertarget{poli-umum-visite-dan-rujukan}{%
\chapter{Poli Umum, Visite dan Rujukan}\label{poli-umum-visite-dan-rujukan}}

\hypertarget{poli-umum}{%
\section{Poli Umum}\label{poli-umum}}

Poli Umum merupakan layanan rawat jalan bagi pasien yang ingin melakukan pemeriksaan general (non spesialistik) atau pemeriksaan penunjang khusus tanpa membawa pengantar pemeriksaan. Pasien yang ingin melakukan pemeriksaan general biasanya tidak mengetahui gejalanya mengarah ke penyakit mana, maka memerlukan pemeriksaan dokter umum. Pasien yang ingin melakukan check-up dengan pemeriksaan penunjang tanpa membawa pengantar sebelumnya dapat melalui poli umum dan berkonsultasi dengan dokter umum. Dokter umum di poli umum diisi oleh dokter jaga. Petugas poli akan menghubungi dokter jaga jika ada pasien umum daftar di loket poli lantai 1. Semua pembiayaan di poli umum ditanggung pasien/asuransi swasta.

Pemeriksaan dilakukan berdasarkan setting pemeriksaan rawat jalan. Pemeriksaan akan didampingi oleh 1-2 orang perawat poliklinik. Pemeriksaan dapat dilakukan di Lt 1 dengan menggunakan ruang poli yang kosong, atau di UGD. Dokter dapat memberikan saran pemeriksaan penunjang yang tersedia di RS dengan persetujuan pasien. Pemeriksaan laboratorum kimiawi serta radiologi non-usg tanpa persiapan khusus hasil dapat diperoleh dalam hitungan 30 menit-jam. Pemeriksaan USG hanya dapat dilakukan pada sore-malam pada hari kerja sehingga dapat diberikan pengantar dan daftar saja untuk USG dan instruksikan pasien datang ke RS pada waktu layanan USG ada di RS. Pemeriksaan radiologi atau lab dengan persiapan khusus diperlukan KIE terkait persiapan dengan baik. Tentukan waktu pasien untuk ke RS kembali dengan membawa pengantar unuk melakukan pemeriksaan setelah persiapan sekiranya terpenuhi.

\hypertarget{visite}{%
\section{Visite}\label{visite}}

\hypertarget{rujukan}{%
\section{Rujukan}\label{rujukan}}

\hypertarget{alat-perhitungan-sederhana}{%
\chapter{Alat Perhitungan Sederhana}\label{alat-perhitungan-sederhana}}

Pada bagian ini akan diisi dengan perhitungan-perhitungan sederhana yang diperlukan dalam penanganan pasien.

\hypertarget{general}{%
\section{General}\label{general}}

\begin{center}\rule{0.5\linewidth}{0.5pt}\end{center}

\hypertarget{penyakit-dalam}{%
\section{Penyakit Dalam}\label{penyakit-dalam}}

\begin{center}\rule{0.5\linewidth}{0.5pt}\end{center}

\hypertarget{pediatri}{%
\section{Pediatri}\label{pediatri}}

\begin{center}\rule{0.5\linewidth}{0.5pt}\end{center}

\hypertarget{kardiovaskular}{%
\section{Kardiovaskular}\label{kardiovaskular}}

\begin{center}\rule{0.5\linewidth}{0.5pt}\end{center}

\hypertarget{bedah}{%
\section{Bedah}\label{bedah}}

\begin{center}\rule{0.5\linewidth}{0.5pt}\end{center}

\hypertarget{anestesi}{%
\section{Anestesi}\label{anestesi}}

\begin{center}\rule{0.5\linewidth}{0.5pt}\end{center}

\hypertarget{neurologi}{%
\section{Neurologi}\label{neurologi}}

\begin{center}\rule{0.5\linewidth}{0.5pt}\end{center}

\hypertarget{obstetri-gineklogi}{%
\section{Obstetri \& Gineklogi}\label{obstetri-gineklogi}}

\hypertarget{terapi-tipikal-di-ugd}{%
\chapter{Terapi Tipikal di UGD}\label{terapi-tipikal-di-ugd}}

Pada bagian ini akan diisi dengan pola tatalaksana sering diberikan pada pasien di UGD. Tatalaksana dikelompokkan berdasarkan kategori kasus. Dosis obat dibuat berdasarkan pengalaman penyusun atas dasar pengetahuan dan instruksi dokter spesialis.

\hypertarget{ugd---rawat-jalan-general-practitioner}{%
\section{UGD - Rawat Jalan (General Practitioner)}\label{ugd---rawat-jalan-general-practitioner}}

\begin{center}\rule{0.5\linewidth}{0.5pt}\end{center}

\hypertarget{penyakit-dalam-1}{%
\section{Penyakit Dalam}\label{penyakit-dalam-1}}

\hypertarget{endokrinologi}{%
\subsection{Endokrinologi}\label{endokrinologi}}

\hypertarget{hiperglikemia-preoperasi}{%
\subsubsection{Hiperglikemia preoperasi}\label{hiperglikemia-preoperasi}}

\hypertarget{ketoasidosis-diabetikum-kad}{%
\subsubsection{Ketoasidosis Diabetikum (KAD)}\label{ketoasidosis-diabetikum-kad}}

\hypertarget{hyperglicemic-hyperosmolar-state}{%
\subsubsection{Hyperglicemic Hyperosmolar State}\label{hyperglicemic-hyperosmolar-state}}

\hypertarget{nefrologi}{%
\subsection{Nefrologi}\label{nefrologi}}

\hypertarget{hiperkalemia}{%
\subsubsection{Hiperkalemia}\label{hiperkalemia}}

\hypertarget{hipokalemia}{%
\subsubsection{Hipokalemia}\label{hipokalemia}}

\hypertarget{hipernatremia}{%
\subsubsection{Hipernatremia}\label{hipernatremia}}

\hypertarget{hiponatremia}{%
\subsubsection{Hiponatremia}\label{hiponatremia}}

\hypertarget{infeksi}{%
\subsection{Infeksi}\label{infeksi}}

\hypertarget{pulmonologi}{%
\subsection{Pulmonologi}\label{pulmonologi}}

\hypertarget{gastroenterologi-hepatologi}{%
\subsection{Gastroenterologi \& Hepatologi}\label{gastroenterologi-hepatologi}}

\hypertarget{kardiovaskular-1}{%
\subsection{Kardiovaskular}\label{kardiovaskular-1}}

\hypertarget{imunologi}{%
\subsection{Imunologi}\label{imunologi}}

\begin{center}\rule{0.5\linewidth}{0.5pt}\end{center}

\hypertarget{paru}{%
\section{Paru}\label{paru}}

\hypertarget{infeksi-1}{%
\subsection{Infeksi}\label{infeksi-1}}

\hypertarget{tuberculosis}{%
\subsubsection{Tuberculosis}\label{tuberculosis}}

\hypertarget{covid-19}{%
\subsubsection{COVID-19}\label{covid-19}}

\hypertarget{pneumonia-bacterial}{%
\subsubsection{Pneumonia Bacterial}\label{pneumonia-bacterial}}

\hypertarget{obstruksi}{%
\subsection{Obstruksi}\label{obstruksi}}

\hypertarget{ppok}{%
\subsubsection{PPOK}\label{ppok}}

\hypertarget{asthma}{%
\subsubsection{Asthma}\label{asthma}}

\hypertarget{malignancy}{%
\subsection{Malignancy}\label{malignancy}}

\hypertarget{lain-lain}{%
\subsection{Lain-lain}\label{lain-lain}}

\hypertarget{efusi-pleura}{%
\subsubsection{Efusi Pleura}\label{efusi-pleura}}

\begin{center}\rule{0.5\linewidth}{0.5pt}\end{center}

\hypertarget{pediatri-1}{%
\section{Pediatri}\label{pediatri-1}}

\hypertarget{neonatologi}{%
\subsection{Neonatologi}\label{neonatologi}}

\hypertarget{infeksi-2}{%
\subsection{Infeksi}\label{infeksi-2}}

\hypertarget{pulmonologi-1}{%
\subsection{Pulmonologi}\label{pulmonologi-1}}

\hypertarget{gastroenterologi-hepatologi-1}{%
\subsection{Gastroenterologi \& Hepatologi}\label{gastroenterologi-hepatologi-1}}

\begin{center}\rule{0.5\linewidth}{0.5pt}\end{center}

\hypertarget{kardiovaskular-2}{%
\section{Kardiovaskular}\label{kardiovaskular-2}}

\hypertarget{pre-operative}{%
\subsection{Pre-operative}\label{pre-operative}}

\hypertarget{iskemia}{%
\subsection{Iskemia}\label{iskemia}}

\hypertarget{disritmiaaritmia}{%
\subsection{Disritmia/Aritmia}\label{disritmiaaritmia}}

\hypertarget{heart-failure}{%
\subsection{Heart Failure}\label{heart-failure}}

\hypertarget{vaskular-perifer}{%
\subsection{Vaskular Perifer}\label{vaskular-perifer}}

\begin{center}\rule{0.5\linewidth}{0.5pt}\end{center}

\hypertarget{bedah-umum-1}{%
\section{Bedah Umum}\label{bedah-umum-1}}

\begin{center}\rule{0.5\linewidth}{0.5pt}\end{center}

\hypertarget{bedah-digestif-1}{%
\section{Bedah Digestif}\label{bedah-digestif-1}}

\hypertarget{obstruksi-1}{%
\subsection{Obstruksi}\label{obstruksi-1}}

\begin{center}\rule{0.5\linewidth}{0.5pt}\end{center}

\hypertarget{bedah-urologi}{%
\section{Bedah Urologi}\label{bedah-urologi}}

\hypertarget{infeksi-3}{%
\subsection{Infeksi}\label{infeksi-3}}

\hypertarget{obstruksi-2}{%
\subsection{Obstruksi}\label{obstruksi-2}}

\hypertarget{malignancy-1}{%
\subsection{Malignancy}\label{malignancy-1}}

\begin{center}\rule{0.5\linewidth}{0.5pt}\end{center}

\hypertarget{bedah-ortopedi}{%
\section{Bedah Ortopedi}\label{bedah-ortopedi}}

\hypertarget{preoperative}{%
\subsection{Preoperative}\label{preoperative}}

\hypertarget{trauma}{%
\subsection{Trauma}\label{trauma}}

\hypertarget{malignancy-2}{%
\subsection{Malignancy}\label{malignancy-2}}

\begin{center}\rule{0.5\linewidth}{0.5pt}\end{center}

\hypertarget{anestesi-1}{%
\section{Anestesi}\label{anestesi-1}}

\hypertarget{preoperative-1}{%
\subsection{Preoperative}\label{preoperative-1}}

\hypertarget{critical-care}{%
\subsection{Critical Care}\label{critical-care}}

\begin{center}\rule{0.5\linewidth}{0.5pt}\end{center}

\hypertarget{neurologi-bedah-saraf}{%
\section{Neurologi \& Bedah Saraf}\label{neurologi-bedah-saraf}}

\hypertarget{vaskular}{%
\subsection{Vaskular}\label{vaskular}}

\hypertarget{pain}{%
\subsection{Pain}\label{pain}}

\hypertarget{trauma-1}{%
\subsection{Trauma}\label{trauma-1}}

\hypertarget{malignancy-3}{%
\subsection{Malignancy}\label{malignancy-3}}

\hypertarget{degenerative}{%
\subsection{Degenerative}\label{degenerative}}

\begin{center}\rule{0.5\linewidth}{0.5pt}\end{center}

\hypertarget{obstetri-gineklogi-1}{%
\section{Obstetri \& Gineklogi}\label{obstetri-gineklogi-1}}

\hypertarget{obstetri}{%
\subsection{Obstetri}\label{obstetri}}

\hypertarget{ginekologi}{%
\subsection{Ginekologi}\label{ginekologi}}

\begin{center}\rule{0.5\linewidth}{0.5pt}\end{center}

\hypertarget{dermatologi-venereologi}{%
\section{Dermatologi \& Venereologi}\label{dermatologi-venereologi}}

\hypertarget{psoriasis}{%
\subsection{Psoriasis}\label{psoriasis}}

\hypertarget{alergi}{%
\subsection{Alergi}\label{alergi}}

\begin{center}\rule{0.5\linewidth}{0.5pt}\end{center}

\hypertarget{tht}{%
\section{THT}\label{tht}}

\hypertarget{preoperaitve}{%
\subsection{Preoperaitve}\label{preoperaitve}}

\hypertarget{corpus-alienum}{%
\subsection{Corpus Alienum}\label{corpus-alienum}}

\hypertarget{infeksi-4}{%
\subsection{Infeksi}\label{infeksi-4}}

\hypertarget{tumor}{%
\subsection{Tumor}\label{tumor}}

\begin{center}\rule{0.5\linewidth}{0.5pt}\end{center}

\hypertarget{mata}{%
\section{Mata}\label{mata}}

\hypertarget{preoperative-2}{%
\subsection{Preoperative}\label{preoperative-2}}

\hypertarget{infeksi-5}{%
\subsection{Infeksi}\label{infeksi-5}}

\hypertarget{corpus-alienum-1}{%
\subsection{Corpus Alienum}\label{corpus-alienum-1}}

\hypertarget{trauma-2}{%
\subsection{Trauma}\label{trauma-2}}

\begin{center}\rule{0.5\linewidth}{0.5pt}\end{center}

\hypertarget{psikiatri}{%
\section{Psikiatri}\label{psikiatri}}

\hypertarget{raptus-mengamuk}{%
\subsection{Raptus (mengamuk)}\label{raptus-mengamuk}}

\begin{center}\rule{0.5\linewidth}{0.5pt}\end{center}

\hypertarget{persiapan-pemeriksaan-penunjang}{%
\section{Persiapan Pemeriksaan Penunjang}\label{persiapan-pemeriksaan-penunjang}}

\hypertarget{radiologi}{%
\subsection{Radiologi}\label{radiologi}}

\hypertarget{usg-abdomen-atas-bawah}{%
\subsubsection{USG Abdomen Atas Bawah}\label{usg-abdomen-atas-bawah}}

\hypertarget{usg-urologi}{%
\subsubsection{USG Urologi}\label{usg-urologi}}

\hypertarget{ct-stonografi}{%
\subsubsection{CT Stonografi}\label{ct-stonografi}}

\hypertarget{laboratorium}{%
\subsection{Laboratorium}\label{laboratorium}}

\hypertarget{asam-urat}{%
\subsubsection{Asam Urat}\label{asam-urat}}

\hypertarget{lipid-profil}{%
\subsubsection{Lipid Profil}\label{lipid-profil}}

\begin{center}\rule{0.5\linewidth}{0.5pt}\end{center}

\hypertarget{informasi-obat-esensial-ugd}{%
\chapter{Informasi Obat Esensial UGD}\label{informasi-obat-esensial-ugd}}

Pada bagian ini akan diisi dengan obat-obat rutin yang sering digunakan dalam penanganan pasien di UGD dikelompokkan berdasarkan kasus. Dosis obat dibuat berdasarkan pengalaman penyusun atas dasar pengetahuan dan instruksi dokter spesialis.

\hypertarget{general-1}{%
\section{General}\label{general-1}}

\hypertarget{sistem-perhitungan-obat-parenteral}{%
\subsection{Sistem Perhitungan Obat Parenteral}\label{sistem-perhitungan-obat-parenteral}}

Rumur Perhitungan Laju Syringe Pump

\textbf{Rumus 1}

\begin{equation}
\\
\frac{Dosis \times BB \times 60}{konsentrasi (mikrogram/cc)}
\\
\end{equation}

\textbf{Rumus 2}

\begin{equation}
\\
\frac{Dosis \times BB \times 60 \times Pengenceran(50 cc)}{Jumlah_obat (mcg)}
\end{equation}

Contoh Kasus:

\begin{itemize}
\tightlist
\item
  Obat untuk pasien berat badan 50 kg
\item
  Dosis awal titrasi vascon 0,1 mcg/kgBB/menit.
\item
  Sedaan Vascon 4 mg per vial.
\end{itemize}

\begin{equation}
\frac{0,1 mcg/kg/min \times 50 kg \times 60}{4000 mcg/50cc}
\\
\frac{5 mcg/min \times 60}{80mcg/cc}
\\ 
\frac{300 mcg/jam}{80 mcg/cc}
\\ 
\ 3,75 cc/jam
\\
\end{equation}

\begin{center}\rule{0.5\linewidth}{0.5pt}\end{center}

\hypertarget{penyakit-dalam-2}{%
\section{Penyakit Dalam}\label{penyakit-dalam-2}}

\hypertarget{parenteral}{%
\subsection{Parenteral}\label{parenteral}}

\hypertarget{transfusi-darah}{%
\subsubsection{Transfusi Darah}\label{transfusi-darah}}

\hypertarget{wb}{%
\paragraph{WB}\label{wb}}

\hypertarget{prc}{%
\paragraph{PRC}\label{prc}}

\hypertarget{ffp}{%
\paragraph{FFP}\label{ffp}}

\hypertarget{tc}{%
\paragraph{TC}\label{tc}}

\hypertarget{albumin}{%
\subsubsection{Albumin}\label{albumin}}

\hypertarget{deksametason}{%
\subsubsection{Deksametason}\label{deksametason}}

\hypertarget{insulin}{%
\subsubsection{Insulin}\label{insulin}}

Sediaan:
* Rapid Acting (cnth: Novorapid): Flexpen isi 3 ml, tiap ml isi 100 unit (Total 300 Unit).
* Short Acting
* Intermediate Acting
* Long Acting

\hypertarget{drip-kad-hhs}{%
\paragraph{DRIP KAD HHS}\label{drip-kad-hhs}}

\hypertarget{regulasi-preoperasi}{%
\paragraph{Regulasi Preoperasi}\label{regulasi-preoperasi}}

\hypertarget{hiperkalemia-1}{%
\paragraph{Hiperkalemia}\label{hiperkalemia-1}}

\hypertarget{dosis-maintenance}{%
\paragraph{Dosis Maintenance}\label{dosis-maintenance}}

\hypertarget{kcl}{%
\subsubsection{KCL}\label{kcl}}

\hypertarget{natrium-bikarbonat}{%
\subsubsection{Natrium Bikarbonat}\label{natrium-bikarbonat}}

\hypertarget{oral}{%
\subsection{Oral}\label{oral}}

\begin{center}\rule{0.5\linewidth}{0.5pt}\end{center}

\hypertarget{paru-1}{%
\section{Paru}\label{paru-1}}

\hypertarget{parenteral-1}{%
\subsection{Parenteral}\label{parenteral-1}}

\hypertarget{aminofilin}{%
\subsubsection{Aminofilin}\label{aminofilin}}

\hypertarget{oral-1}{%
\subsection{Oral}\label{oral-1}}

\begin{center}\rule{0.5\linewidth}{0.5pt}\end{center}

\hypertarget{pediatri-2}{%
\section{Pediatri}\label{pediatri-2}}

\begin{center}\rule{0.5\linewidth}{0.5pt}\end{center}

\hypertarget{kardiovaskular-3}{%
\section{Kardiovaskular}\label{kardiovaskular-3}}

\hypertarget{parenteral-2}{%
\subsection{Parenteral}\label{parenteral-2}}

\hypertarget{dopamin}{%
\subsubsection{Dopamin}\label{dopamin}}

\begin{itemize}
\tightlist
\item
  sediaan 1 ampul = 200 mg
\end{itemize}

\begin{longtable}[]{@{}lll@{}}
\toprule
\begin{minipage}[b]{(\columnwidth - 2\tabcolsep) * \real{0.34}}\raggedright
Kategori\strut
\end{minipage} & \begin{minipage}[b]{(\columnwidth - 2\tabcolsep) * \real{0.24}}\raggedright
Dosis\strut
\end{minipage} & \begin{minipage}[b]{(\columnwidth - 2\tabcolsep) * \real{0.41}}\raggedright
Keterangan\strut
\end{minipage}\tabularnewline
\midrule
\endhead
\begin{minipage}[t]{(\columnwidth - 2\tabcolsep) * \real{0.34}}\raggedright
Rendah\strut
\end{minipage} & \begin{minipage}[t]{(\columnwidth - 2\tabcolsep) * \real{0.24}}\raggedright
1-5 mcg/kgBB/menit\strut
\end{minipage} & \begin{minipage}[t]{(\columnwidth - 2\tabcolsep) * \real{0.41}}\raggedright
Reseptor dopaminergik terutama di ginjal, mesenterium dan pembuluh koroner\strut
\end{minipage}\tabularnewline
\begin{minipage}[t]{(\columnwidth - 2\tabcolsep) * \real{0.34}}\raggedright
Sedang\strut
\end{minipage} & \begin{minipage}[t]{(\columnwidth - 2\tabcolsep) * \real{0.24}}\raggedright
5-10 mcg/kgBB/Menit\strut
\end{minipage} & \begin{minipage}[t]{(\columnwidth - 2\tabcolsep) * \real{0.41}}\raggedright
Meningkatnya tekanan sistolik dan tekanan nadi tanpa mengubah tekanan diastolik\strut
\end{minipage}\tabularnewline
\begin{minipage}[t]{(\columnwidth - 2\tabcolsep) * \real{0.34}}\raggedright
Tinggi\strut
\end{minipage} & \begin{minipage}[t]{(\columnwidth - 2\tabcolsep) * \real{0.24}}\raggedright
10 - 20 mcg/kgBB/menit\strut
\end{minipage} & \begin{minipage}[t]{(\columnwidth - 2\tabcolsep) * \real{0.41}}\raggedright
Vasopressor\strut
\end{minipage}\tabularnewline
\bottomrule
\end{longtable}

Kontraindikasi:

\begin{itemize}
\tightlist
\item
  Hipovolemik belum terkoreksi
\item
  Takiartimia/fibrilasi ventrikel
\item
  Hipertiroid
\end{itemize}

\hypertarget{dobutamin}{%
\subsubsection{Dobutamin}\label{dobutamin}}

\begin{itemize}
\tightlist
\item
  Sediaan ampul = 250 mg
\end{itemize}

\begin{longtable}[]{@{}lll@{}}
\toprule
kategori & Dosis & Keterangan\tabularnewline
\midrule
\endhead
rendah & 2-5 mcg/kgBB/menit &\tabularnewline
Sedang & 5-10 mcg/kgBB/Menit &\tabularnewline
Tinggi & 10-20 mcg/kgBB/menit &\tabularnewline
\bottomrule
\end{longtable}

Kontraindikasi:

\begin{itemize}
\tightlist
\item
  Stenosis subaortik
\item
  Hipertropik Idiopatik
\end{itemize}

\hypertarget{norepinefrinvascon}{%
\subsubsection{Norepinefrin/Vascon}\label{norepinefrinvascon}}

Sediaan 1 ampul = 4 mg

\begin{longtable}[]{@{}lll@{}}
\toprule
Kategori & Dosis & Keterangan\tabularnewline
\midrule
\endhead
& 0,1 - 0,5 mcg/kgBB/menit &\tabularnewline
\bottomrule
\end{longtable}

\hypertarget{nicardipin}{%
\subsubsection{Nicardipin}\label{nicardipin}}

\begin{itemize}
\tightlist
\item
  Sediaan: 10 mg/ampul
\item
\end{itemize}

\hypertarget{isdn}{%
\subsubsection{ISDN}\label{isdn}}

\begin{itemize}
\tightlist
\item
  Sediaan: 1 ampul 10 mg
\item
  Dosis 1-10 mg/jam
\item
  Biasanya diencerkan 2 ampul
\end{itemize}

\hypertarget{nitrogliserin}{%
\subsubsection{Nitrogliserin}\label{nitrogliserin}}

\begin{itemize}
\tightlist
\item
  Sediaan = 50 mg (ampul)
\end{itemize}

\hypertarget{furosemid}{%
\subsubsection{Furosemid}\label{furosemid}}

\begin{itemize}
\tightlist
\item
  Sediaan = 20 mg/2cc (ampul)
\item
  Dosis
\end{itemize}

\hypertarget{heparin}{%
\subsubsection{Heparin}\label{heparin}}

\begin{itemize}
\tightlist
\item
  Sediaan : 25.000 Unit/vial
\item
  Obat dimasukan di 500 cc NS
\item
  Perhitungan kecepatan tetesan infus sesuai aturan tetes infus
\item
  Dosis:
\end{itemize}

\hypertarget{lmwh}{%
\subsubsection{LMWH}\label{lmwh}}

\hypertarget{digoxin}{%
\subsubsection{Digoxin}\label{digoxin}}

\hypertarget{adrenalinepinefrin}{%
\subsubsection{Adrenalin/Epinefrin}\label{adrenalinepinefrin}}

\hypertarget{sulfas-atropin}{%
\subsubsection{Sulfas Atropin}\label{sulfas-atropin}}

\hypertarget{amiodaron}{%
\subsubsection{Amiodaron}\label{amiodaron}}

\hypertarget{lidokain}{%
\subsubsection{Lidokain}\label{lidokain}}

\hypertarget{calcium-glukonas}{%
\subsubsection{Calcium Glukonas}\label{calcium-glukonas}}

\hypertarget{oral-nitrogliserin}{%
\subsection{Oral Nitrogliserin}\label{oral-nitrogliserin}}

\begin{center}\rule{0.5\linewidth}{0.5pt}\end{center}

\hypertarget{bedah-1}{%
\section{Bedah}\label{bedah-1}}

\begin{center}\rule{0.5\linewidth}{0.5pt}\end{center}

\hypertarget{anestesi-2}{%
\section{Anestesi}\label{anestesi-2}}

\hypertarget{analgetik}{%
\subsection{Analgetik}\label{analgetik}}

\hypertarget{morfin}{%
\subsubsection{Morfin}\label{morfin}}

\hypertarget{petidin}{%
\subsubsection{Petidin}\label{petidin}}

\hypertarget{fentanil}{%
\subsubsection{Fentanil}\label{fentanil}}

\hypertarget{ketorolac}{%
\subsubsection{Ketorolac}\label{ketorolac}}

\hypertarget{tramadol}{%
\subsubsection{Tramadol}\label{tramadol}}

\hypertarget{sedatif}{%
\subsection{Sedatif}\label{sedatif}}

\hypertarget{diazepam}{%
\subsubsection{Diazepam}\label{diazepam}}

\hypertarget{midazolam}{%
\subsubsection{Midazolam}\label{midazolam}}

\hypertarget{propofol}{%
\subsubsection{Propofol}\label{propofol}}

\hypertarget{muscle-relaxant}{%
\subsection{Muscle Relaxant}\label{muscle-relaxant}}

\begin{center}\rule{0.5\linewidth}{0.5pt}\end{center}

\hypertarget{neurologi-neurosurgery}{%
\section{Neurologi \& Neurosurgery}\label{neurologi-neurosurgery}}

\hypertarget{parenteral-3}{%
\subsection{Parenteral}\label{parenteral-3}}

\hypertarget{manitol}{%
\subsubsection{Manitol}\label{manitol}}

\hypertarget{nacl-3}\label{nacl-3}}

\hypertarget{fenitoin}{%
\subsubsection{Fenitoin}\label{fenitoin}}

\hypertarget{fenobarbital}{%
\subsubsection{Fenobarbital}\label{fenobarbital}}

\hypertarget{citicolin}{%
\subsubsection{Citicolin}\label{citicolin}}

\hypertarget{piracetam}{%
\subsubsection{Piracetam}\label{piracetam}}

\hypertarget{oral-2}{%
\subsection{Oral}\label{oral-2}}

\begin{center}\rule{0.5\linewidth}{0.5pt}\end{center}

\hypertarget{obstetri-gineklogi-2}{%
\section{Obstetri \& Gineklogi}\label{obstetri-gineklogi-2}}

\hypertarget{parenteral-4}{%
\subsection{Parenteral}\label{parenteral-4}}

\hypertarget{mgso4}{%
\subsubsection{MgSO4}\label{mgso4}}

\hypertarget{oral-3}{%
\subsection{Oral}\label{oral-3}}

\begin{center}\rule{0.5\linewidth}{0.5pt}\end{center}

\hypertarget{dermatologi-venereologi-1}{%
\section{Dermatologi \& Venereologi}\label{dermatologi-venereologi-1}}

\begin{center}\rule{0.5\linewidth}{0.5pt}\end{center}

\hypertarget{tht-1}{%
\section{THT}\label{tht-1}}

\begin{center}\rule{0.5\linewidth}{0.5pt}\end{center}

\hypertarget{mata-1}{%
\section{Mata}\label{mata-1}}

\begin{center}\rule{0.5\linewidth}{0.5pt}\end{center}

\hypertarget{hasil-rapat}{%
\chapter{Hasil Rapat}\label{hasil-rapat}}

\hypertarget{section}{%
\section{2021}\label{section}}

\begin{center}\rule{0.5\linewidth}{0.5pt}\end{center}

\hypertarget{september}{%
\subsection{September}\label{september}}

\begin{center}\rule{0.5\linewidth}{0.5pt}\end{center}

\hypertarget{oktober}{%
\subsection{Oktober}\label{oktober}}

\begin{center}\rule{0.5\linewidth}{0.5pt}\end{center}

\hypertarget{november}{%
\subsection{November}\label{november}}

Waktu: 26/11/2021

\begin{center}\rule{0.5\linewidth}{0.5pt}\end{center}

\hypertarget{desember}{%
\subsection{Desember}\label{desember}}

\begin{center}\rule{0.5\linewidth}{0.5pt}\end{center}

\hypertarget{section-1}{%
\section{2022}\label{section-1}}

\begin{center}\rule{0.5\linewidth}{0.5pt}\end{center}

\hypertarget{januari}{%
\subsection{Januari}\label{januari}}

\begin{center}\rule{0.5\linewidth}{0.5pt}\end{center}

\hypertarget{februari}{%
\subsection{Februari}\label{februari}}

\begin{center}\rule{0.5\linewidth}{0.5pt}\end{center}

\hypertarget{maret}{%
\subsection{Maret}\label{maret}}

\begin{center}\rule{0.5\linewidth}{0.5pt}\end{center}

\hypertarget{april}{%
\subsection{April}\label{april}}

\begin{center}\rule{0.5\linewidth}{0.5pt}\end{center}

\hypertarget{mei}{%
\subsection{Mei}\label{mei}}

\begin{center}\rule{0.5\linewidth}{0.5pt}\end{center}

\hypertarget{juni}{%
\subsection{Juni}\label{juni}}

\begin{center}\rule{0.5\linewidth}{0.5pt}\end{center}

\hypertarget{juli}{%
\subsection{Juli}\label{juli}}

\begin{center}\rule{0.5\linewidth}{0.5pt}\end{center}

\hypertarget{agustus}{%
\subsection{Agustus}\label{agustus}}

\begin{center}\rule{0.5\linewidth}{0.5pt}\end{center}

  \bibliography{book.bib,packages.bib}

\end{document}
